%!TEX root = ../../main.tex
\section{NFT-Related systems}
As mentioned before, the \emp{Ethereum Foundation} was first on applying the concept of \emp{Smart Contracts} over the Blockchain. When this happened a set of application-level standards, name registries, library/package formats, programming language and all the structure of the framework were officially documented in a repository known as \ac{ERC} in 2019. In this site many guidelines known as \ac{EIP} can be found on how to create code for different smart contract specifications. For this work project, the \ac{EIP}s consulted were:
\begin{itemize}
    \item ERC-20. Token Standard. Describes the implementation of a standard API for tokens within smart contracts.\cite{EIP20Tok46:online}
    \item ERC-721. \ac{NFT} Token Standard. The following standard allows for the implementation of a standard API for NFTs within smart contracts. This standard provides basic functionality to track and transfer \ac{NFT}s\cite{EIP721No36:online}.
    \item ERC-115. Multi Token Standard. A standard interface for contracts that manage multiple token types. A single deployed contract may include any combination of fungible tokens, non-fungible tokens or other configurations (e.g. semi-fungible tokens)\cite{EIP1155M92:online}.
\end{itemize}

Since its creation, the \ac{NFT} Standard, has been used for the \ac{DAO}s and systems have been created to provide tools to final users for the usage of unique assets. More of this work however has been implemented in the Open Blockchains.

\subsection{NFT on Open Blockchains}
Many initiatives have emerged as prototypes to explain to the general public the importance, relevance, and potential of NFT for storing and certifying digital unique assets. At the beginning pure art concepts or games on the \emp{Ethereum} Blockchain have been very successful, and led to the generation of speculative markets with high volatility and fraud due to its economical and lucrative basis \cite{TheHisto62:online}.

The most successful NFT projects speaking in terms of usability and profitability are:
\begin{itemize}
    \item \textbf{Rare Pepes (2017)}. With Ethereum gaining prominence in early 2017, memes started to be traded there as well. A project named "Peperium" was announced to be a “decentralized meme marketplace and \ac{TCG} that allowed anyone to create memes that live eternally on IPFS and Ethereum.”
    \item \textbf{Cryptopunks (2017)}. A first set of 10,000 unique computer-generated characters on the \emph{Ethereum} Blockchain. The platform opted to let anyone with an Ethereum wallet claim a Cryptopunk for free.
    
    All 10,000 Cryptopunks were swiftly claimed and started a thriving secondary marketplace where people bought and sold them. By the time they were created no \ac{ERC}-721 standard existed.
    \item \textbf{CryptoKitties (2017)}. It was the first NFT implementation to come mainstream. It is a blockchain-based virtual game that allows players to adopt, raise, and trade virtual cats with a unique genetic code embedded in the smart contract. No single asset would evolve or have same characteristics as the other and they would show new characteristics over time. 
    \item \textbf{OpenSea (2017)}. Is an American online \ac{NFT} marketplace aimed to work as a trading market for digital art and buyers. It is the most famous platform of its kind.
\end{itemize}

\subsection{Permissioned Blockchains}
Contrary to permissionless NFT platforms, permissioned Blockchains allow use of corporate data and business workflows to operate inter-system and inter-company wise.

Examples of NFT Permissioned Platforms are:
\begin{itemize}
    \item  \textbf{A Decentralized Framework for Patents and
Intellectual Property as NFT in Blockchain Networks}\cite{bamakan2021decentralized}. It proposes a system to implement store of intellectual property and patents in a corporate multi-shared NFT Blockchain platform. It proposes the architecture of a system with decentralized authentication and decentralize storage.
    \item \textbf{Design of extensible non-fungible token model in Hyperledger fabric}\cite{hong2019design}
    Presents an extensible \ac{NFT} token model for supporting such assets in Hyperledger Fabric It also applies extensible NFTs such as document and signature tokens to a decentralized signature service.
\end{itemize}